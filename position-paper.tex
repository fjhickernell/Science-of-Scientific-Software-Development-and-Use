\documentclass{amsart}
\usepackage{mathtools,upref,siunitx,upquote,fancyvrb,xspace,wrapfig,color}
\usepackage[hyphens]{url}
\usepackage[utf8]{inputenc}

\input{FJHDef.tex}

\textwidth 6.5in
\hoffset -0.7in
\textheight 9in
\voffset -0.5in
\setlength{\parskip}{0em}
\usepackage[compact]{titlesec}
    \titlespacing{\section}{0pt}{1ex}{0ex}
    \titlespacing{\subsection}{0pt}{1ex}{\wordsep}
    \titleformat{\section}[block]{\scshape \fillast}{\thesection.}{0.5ex}{}
    \titleformat{\subsection}[runin]{\normalfont \bfseries}{\normalfont\thesubsection.}{0.5ex}{}[.]

\usepackage{algpseudocode}
\usepackage{algorithm, algorithmicx}
\algnewcommand\algorithmicparam{\textbf{Parameters:}}
\algnewcommand\PARAM{\item[\algorithmicparam]}
\algnewcommand\algorithmicinput{\textbf{Input:}}
\algnewcommand\INPUT{\item[\algorithmicinput]}
\algnewcommand\RETURN{\State \textbf{Return }}

\iffalse
Important Dates and Links

19th November 2021: Deadline for position paper submission
3rd December 2021: Notification of position acceptance
13 - 15th December 2021: Workshop
SUBMISSION URL: https://orausurvey.orau.org/n/SSSDU.aspx
WORKSHOP URL: https://www.orau.gov/SSSDU2021
Motivation

Software is an increasingly important component in the pursuit of scientific discovery. Both its development and use are essential activities for many scientific teams. At the same time, very little scientific study has been conducted to understand, characterize, and improve the development and use of software for science.

Computational science teams have diversified over time to include contributions from domain scientists who provide expertise in scientific and engineering disciplines, applied mathematicians and computer scientists who provide optimal algorithms and data structures, and software and data engineers who provide methodologies and tools adapted and adopted from other software domains. These diverse contributions have enabled tremendous advances in the pursuit of scientific discovery, even as models, computer architectures, and software environments have become more complicated.

With this increasing diversity, we believe the next opportunity for qualitative improvement comes from applying the scientific method to understanding, characterizing, and improving how scientific software is developed and used. We believe that this pursuit requires expertise from computational scientists themselves, and from the cognitive and social sciences as well as the software engineering research community.

As we look to increase the productivity and sustainability of the scientific-software- development-and-use cycle, a more systematic application of the scientific method to understand processes for software development and use [1,2] will be a valuable tool to guide future work and result in more usable and sustainable software. This workshop will bring together computer scientists, software engineering researchers, computational scientists, applied mathematicians, social scientists, cognitive scientists, and others, to explore how we can conduct such systematic investigations, what can be learned, and how doing so will benefit the scientific enterprise.

The workshop will be structured around a set of breakout sessions, with every attendee expected to participate actively in the discussions. Afterward, workshop attendees — from DOE, industry, and academia — will produce a report for ASCR that summarizes the findings of the workshop.

[1] https://bssw.io/blog_posts/research-software-science-a-scientific-approach-to-understanding-and-improving-how-we-develop-and-use-software-for-research
[2] https://www.exascaleproject.org/better-scientific-productivity-through-better-scientific-software-the-ideas-report/

Invitation

We invite community input in the form of two-page position papers that identify and discuss key challenges and opportunities in the science of the scientific-software development process and the study of the use of scientific software. In addition to providing an avenue for identifying workshop participants, these position papers will be used to shape the workshop agenda, identify panelists, and contribute to the workshop report. Position papers should not describe the authors’ current or planned research, contain material that should not be disclosed to the public, recommend specific solutions, or discuss narrowly-focused research topics. Rather, position papers should aim to improve the community’s shared understanding of the problem space, identify challenging research directions, and help to stimulate discussion.

One author of each selected submission will be invited to participate in the workshop.

By submitting a position paper, authors consent to have their position paper published publicly.

Authors are not required to have a history of funding by the ASCR Computer Science program.

Submission Guidelines

Position Paper Structure and Format

Position papers should follow the following format:

Title
Authors (with affiliations and email addresses)
Challenge: Identify limitations of state-of-the-art practice with examples
Opportunity: Describe how the identified challenges may be addressed, whether through new tools and techniques, new technologies, new methodologies, or new groups collaborating in the process
Timeliness or maturity: Why now? What breakthrough or change makes progress possible now where it wasn’t possible previously? What will be the impact of success?
References
Each position paper must be no more than two pages including figures and references. The paper may include any number of authors but contact information for a single author who can represent the position paper at the workshop must be provided with the submission. There is no limit to the number of position papers that an individual or group can submit. Authors are strongly encouraged to follow the structure outlined above. Papers should be submitted in PDF format using the designated page on the workshop website.

Notional Questions

Position papers should present views on why and how scientific-software development and use can be studied as a scientific endeavor, perhaps taking inspiration from some of the following:

What methods can be used to study the development and/or use of scientific software?
What methods either uniquely apply or uniquely do not apply to the development of scientific software?
How can we best integrate social and cognitive sciences into scientific software activities?
How can we study changes in culture affecting scientific software development and use?
What unique changes in scientific software development would improve productivity, accuracy, trust, and/or reliability?
How can we motivate changes in the scientific-computing community to improve practices for software development and use?
How will AI-driven development tools affect best practices for scientific software development and use?
How will static and dynamic analysis methods affect the practice of scientific software development?
How are scientific software developers and users atypical from other larger software communities?
When and how can research and methods from other software communities be adapted and adopted for scientific software?
How is the field of computational science changing, and how will it change?
What are the roadblocks faced by computational scientists in computational application use?
What are the challenges faced by computational scientists in their pursuit of discoveries?
Selection

Submissions will be reviewed by the workshop’s organizing committee using criteria of overall quality, relevance, likelihood of stimulating constructive discussion, and ability to contribute to an informative workshop report. Unique positions that are well presented and emphasize potentially-transformative research directions will be given preference.

Organizing Committee

Mike Heroux, Sandia National Laboratories
David Bernholdt, Oak Ridge National Laboratory
Lois Curfman McInnes, Argonne National Laboratory
John Cary, University of Colorado
Sponsor: Department of Energy, Office of Science, Advanced Scientific Computing Research


\fi

\newcommand{\AGSNote}[1]{{\color{cyan}Aleksei: #1}}
\newcommand{\FJHNote}[1]{{\color{blue}Fred: #1}}
\newcommand{\SCTCNote}[1]{{\color{green}Sou-Cheng: #1}}


\usepackage{biblatex}
\addbibresource{FJH23.bib}
\addbibresource{FJHown23.bib}

\usepackage{url}
\usepackage{hyperref}
\hypersetup{
    colorlinks=true,
    linkcolor=blue,
    filecolor=magenta,      
    urlcolor=blue}

\begin{document}
\title{Position Paper on the \\ Science of Scientific-Software Development and Use}
\author{\vspace{-2ex}Sou-Cheng Terrya Choi}
\address[Choi, Ding, Hickernell, Sorokin]{Department of Applied Mathematics, RE 220, 10 W. 32nd St., Chicago, IL 60616}
\email{schoi32@iit.edu, yding2@iit.edu, hickernell@iit.edu, asorokin@hawk.iit.edu   }
\author{Yuhan Ding}
\author{Claude Hall Jr.,}
\address[Hall]{Birmingham Southern College, 900 Arkadelpha Rd, Birmingham, AL, 35254}
\email{cdhall1@bsc.edu}
\author{Fred J. Hickernell}
\author{Aleksei Sorokin}

%\begin{abstract}This project is where all of the files go that are needed elsewhere
%\end{abstract}

\maketitle

\vspace{-5ex}

The opportunities for discovery, prediction, and optimization provided by advances in computer hardware and software are immense, but the challenges to taking full advantages of these opportunities are substantial.  
%As noted in \cite{ECP2020a}, ``Members of the CSE [computational science and engineering] community \ldots face an urgent need to improve developer productivity, positively impacting product quality, development time, and staffing resources, and software sustainability, reducing the cost of maintaining, sustaining, and evolving software capabilities.'' 
This position paper focuses on four challenges: i) connectivity among software packages, ii) data driven error estimates/bounds that propagate across algorithms, iii) taking advantage of diverse hardware architectures, and education of CSE developers and users.


\section{Challenge} % Identify limitations of state-of-the-art practice with examples
\subsection{Connectivity} 
%There is a multitude of scientific software libraries ranging from hundreds of lines of code to millions of lines of code.  
The developer wishing to develop a new (addition to a) library and the practitioner wishing to utilize multiple libraries together find that basic mathematical concepts may be coded differently in different libraries.  Should a function algorithm include its domain and the number of scalar inputs?  Should an algorithm providing data sites provide a fixed number or an extensible sequence?  What expectations must developers meet so that the pull requests with their new or modified algorithms will be successful? The (quasi-)Monte Carlo and probabilistic numerics libraries with which the authors are most familiar such as \cite{QMCPy2020a,SciPyQMC,Stan,probnum} lack  adequate connectivity.

\subsection{Data driven error estimation} 
Complex computations often require the chaining of algorithms, e.g., the output of a differential equation solver becomes the input of an optimizer.  Since the next algorithm in the chain accepts inexact input from the previous algorithm, we need practical error bounds on algorithm output. A priori algorithm error bounds based on unknown norms of input functions are unhelpful.  Reliable data driven error bounds or estimates are needed. Moreover, error analysis for algorithms needs to include the possibility of inexact inputs.

\FJHNote{Get Ilse's slides}

\subsection{Diverse hardware} \SCTCNote{I will take the first stab.}

\begin{wrapfigure}{r}{0.5\textwidth}
    \includegraphics[width = 0.5\textwidth]{cropped-ideas-watersheds-software-ecosystem-road.jpg}
\end{wrapfigure}

\subsection{Education} 
First courses in numerical methods \cite{BurFaiBur16a} introduce students to the basic numerical algorithms.  Courses in computational science \cite{HolEic19a,ShifShif14a} combine numerical methods with modeling.  Students are rarely taught to combine multiple algorithms together to solve a problem.  This leaves them unprepared for  large scale computational science, which is characterized by  ``increasing demands for predictive multiscale and multiphysics simulations, analysis, and design,'' as explained in \cite{IDEASAbout}.  The IDEAS Watershed project, whose codebase is depicted on the right \cite{IDEASWatershedPict}, is an example.

   







\section{Opportunity} % Describe how the identified challenges may be addressed, whether through new tools and techniques, new technologies, new methodologies, or new groups collaborating in the process
\subsection{Connectivity} 
Code across multiple languages and platforms adheres to standards for certain data types, such as IEEE 754 for floating point arithmetic, \url{https://en.wikipedia.org/wiki/IEEE_754}.  We propose that groups of interested parties connected with larger scientific libraries develop shared standards for other key objects that appear in algorithms, such as functions, sampling sites, and (partial) differential equations. This would promote connectivity among software libraries and allow smaller libraries to be merged into larger ones.  Developers could more easily contribute to existing libraries, and practitioners would find it easier to use multiple libraries.

\subsection{Data driven error estimation} 
Several of the authors and their collaborators have developed data-driven error bounds for integration and univariate function approximation and optimization using numerical analysis and probabilistic numerics approaches \cite{ChoEtal17a, HicEtal14b, HicEtal14a, HicJim16a, RatHic19a,JimHic16a}.  The key behind these error bounds is to identify reasonable cones of input functions that are not too spiky or peaky.  This facilitates theoretical guarantees on the data-driven error bounds.  The ideas referenced here could be extended to other algorithms such as differential equation solvers.  What is also needed is for the error bounds derived to allow for inexact input, which comes from chaining multiple algorithms together.

\subsection{Diverse Hardware} \SCTCNote{I will take the first stab.}

\subsection{Education} \FJHNote{I will take the first stab.} %Claude: I will try to work on this as well

%This will help students learn more about large-scale computing. 


\section{Timeliness} % Why now? What breakthrough or change makes progress possible now where it wasn’t possible previously? What will be the impact of success?

The Extreme-scale Scientific Software Development Kit (xSDK, \url{https://xsdk.info}) is an effort to ``address challenges in interoperability and sustainability of software developed by diverse groups at different institutions.'' The push to identify standards for objects such as functions and partial differential equations, would fit this program.

\printbibliography
%\bibliographystyle{amsplain}
%\bibliography{FJH23,FJHown23}

The challenges of educating researchers to develop shared code

Inconsistent APIs across similar packages

The hurdles of getting algorithms into large packages

The difficulties in correcting algorithms with defects

Problems with scaling algorithms to multi-core, GPU, and multi-node architectures



Error bounds in each step of the pipeline, cones

Ramping up from individual software to larger projects

Connectivity of packages

Definitions that fit the problem of the software 

\AGSNote{
\subsection{Reproducibility}

Another challenge is reproducibility and ease of access to complex, interconnected frameworks. Reproducible code is consistently documented, thoroughly tested, and widely distributed. 

Documentation enables users to quickly learn essential information pertaining to a particular software element. We advocate for documentation in two forms: both online and inline. Online documentation enables users to quickly learn standards of interaction with a given package element.  Inline documentation lowers the barrier to entry for  developers and documents implementation specific nuances. Documentation is challenging due to its' dynamic nature and required synchronization with the codebase. 

Tests for accuracy, consistency, and performance are a key element of any high quality scientific-software development. Developing a testing framework is time consuming and non-trivial due to the often hierarchical testing suite that must be build. In addition, it can be difficult for an organization to justify devoting hours to test development when they are often underappreciated by the end user. However, a robust testing suite is crucial to helping both users and developers operate with confidence within a frameworks parameters. 

Package distribution is the first line of offense for recruiting new users. Oftentimes, high quality packages are underutilized simply because they are not distributed in a standard fashion. Due to the influx of software development on all scales, a package that is difficult to installed will quickly deter potential users. While a users setup experience should be seamless, distribution can be quite the opposite. Some distribution platforms require additional standards that further hinder developers freedom. 
}

\end{document}
